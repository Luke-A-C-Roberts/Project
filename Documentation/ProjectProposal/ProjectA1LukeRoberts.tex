\documentclass[runningheads]{llncs}
\usepackage{graphicx}

\begin{document}

% Comparing the Effectiveness of Deep Learning Models
% on the Categorisation of Galaxy Morphology
\title{CMP3753M Project Proposal}

\author{Luke Roberts}

\institute{University of Lincoln \\
\email{25722923@students.lincoln.ac.uk}}

\maketitle

\section{Introduction}
%   Overview
% - Introduce key concepts like deep learning and
%   image classification, and why they are relevent
%   to study -> What are their uses?
%
% - Introduce galaxy morphology and the importance of
%   further studying distant galaxies for bettering
%   human understanding of the cosmos -> Why is it important?
%
% - Link the two ideas and showcase previous research,
%   adding references when appropriate
%
%   Paragraph 1
% - What is deep learning and image classification
% - Image classification accurately (but not always perfectly)
%   catagorise images in a specific dataset
% - (we will be using a specific dataset of images)
% - Different image classification models work better with
%   different themed image sets
%
%   Paragraph 2
% - Astronomy and understanding the universe is important
%   for scientific research
% - Understanding galaxy forms is a small but important part
%   of astronomy
% - Because of the massive amount of data produced by
%   observatries e.g. the hubble space telescope, it takes
%   a long time for people to accurately catagorise the
%   growing dataset.
% - There exists databases of already catagorised
%   galaxies
% - We should therefore use this data to catagorise new
%   images of galaxy morphology
%
%   Conclusion
% - We should analise which Deep Learning Model is best for
%   catagorizing galaxies into different morphologies.
Understanding the universe has been a pursuit of humanity for thousands of
years. Because of advances in observational technology such as the Hubble Space
Telescope (HST) and more recently the James Webb Space Telescope (JWST), we can
observe very distant galaxies. The time taken for the light of distant galaxies to
reach us means that our night sky is a window into the past, allowing
astrophysicists to understand the evolution of our universe in more depth. Photographs
such as the Hubble Ultra-Deep field show us an ancient universe full of developing
galaxies [], and the amount of observed galaxies is only increasing. By identifying
how galaxy morphology changes over time, we have a clearer picture of the
overall changes over the last several billion years. However, it is extremely
time-consuming for scientists to categorise galaxies in their research.
A study in 2016 \cite{article_1}calculated that there are ~2.0×10¹² galaxies in
the observable universe, and it is therefore reasonable to suggest automating
galaxy categorisation using computers. One way to achieve this is to train a
deep learning model to catagorise images of galaxies.

Deep Learning has recently revolutionised both scientific research and modern
life in a profound way. From the categorisation of X-ray images in the medical
field; machine translation of natural language such as Google Translate and large
text generation models such as ChatGPT, deep learning has become the best way to
categorise, analyse and generate unstructured data. To do many of these tasks, which were
once thought impossible, machine learning engineers create deep learning models
which utilise a dataset that learns patterns about that data. Image
classification is a form of deep learning that uses images as input data and is used for
a wide variety of applications in scientific research. Like all forms of deep learning,
image processing models are trained so that they more accurately categorise new input
data []. At the start of training, the model performs poorly when tested,
however, through analysing how the output fails, the model can tweak its own parameters
in order to achieve greater accuracy. The choice of which machine learning model
to use and the specific hyperparameters are important for maximising the efficiency
of the model. This can be due to the content and quality of the training data being
used.

\section{Aims and Objectives}

\subsection{Aims}

\subsection{Objectives}

\section{Project Plan and Risk Analysis}

\subsection{Project Plan}

\subsection{Risk Analysis}

\begin{thebibliography}{8}

\bibitem{article_1}
Castelvecchi, D. (2016). Universe has ten times more galaxies than researchers thought. Nature. \doi{urlhttps://doi.org/10.1038/nature.2016.20809.}

% \bibitem{article_2}
% Diego, J.A. de, Nadolny, J., Bongiovanni, Á., Cepa, J., Pović, M., García, A.M.P., Torres, C.P.P., Lara-López, M.A., Cerviño, M., Martínez, R.P., Alfaro, E.J., Castañeda, H.O., Fernández-Lorenzo, M., Gallego, J., González, J.J., González-Serrano, J.I., Pintos-Castro, I., Sánchez-Portal, M., Cedrés, B. and González-Otero, M. (2020). Galaxy classification: deep learning on the OTELO and COSMOS databases. Astronomy \& Astrophysics, [online] 638, p.A134. doi:https://doi.org/10.1051/0004-6361/202037697.

\end{thebibliography}

\end{document}