\documentclass[runningheads]{llncs}
\usepackage{graphicx}

\begin{document}

% Comparing the Effectiveness of Deep Learning Models
% on the Categorisation of Galaxy Morphology
\title{CMP3753M Project Proposal}

\author{Luke Roberts}

\institute{University of Lincoln \\
\email{25722923@students.lincoln.ac.uk}}

\maketitle

\section{Introduction}
%   Overview
% - Introduce key concepts like deep learning and
%   image classification, and why they are relevent
%   to study -> What are their uses?
%
% - Introduce galaxy morphology and the importance of
%   further studying distant galaxies for bettering
%   human understanding of the cosmos -> Why is it important?
%
% - Link the two ideas and showcase previous research,
%   adding references when appropriate

%   Paragraph 1
% - What is deep learning and image classification
% - Image classification accurately (but not always perfectly)
%   catagorise images in a specific dataset
% - (we will be using a specific dataset of images)
% - Different image classification models work better with
%   different themed image sets

%   Paragraph 2
% - Astronomy and understanding the universe is important
%   for scientific research
% - Understanding galaxy forms is a small but important part
%   of astronomy
% - Because of the massive amount of data produced by
%   observatries e.g. the hubble space telescope, it takes
%   a long time for people to accurately catagorise the
%   growing dataset.
% - There (therefore) exists databases of already catagorised
%   galaxies
% - We should therefore use this data to catagorise new
%   images of galaxy morphology

%   Conclusion
% - We should analise which Deep Learning Model is best for
%   catagorizing galaxies into different morphologies.

\section{Aims and Objectives}

\subsection{Aims}



\subsection{Objectives}

\section{Project Plan and Risk Analysis}

\subsection{Project Plan}

\subsection{Risk Analysis}

\begin{thebibliography}{8}

\end{thebibliography}

\end{document}